%%
%% This is file `sample-sigplan.tex',
%% generated with the docstrip utility.
%%
%% The original source files were:
%%
%% samples.dtx  (with options: `sigplan')
%% 
%% IMPORTANT NOTICE:
%% 
%% For the copyright see the source file.
%% 
%% Any modified versions of this file must be renamed
%% with new filenames distinct from sample-sigplan.tex.
%% 
%% For distribution of the original source see the terms
%% for copying and modification in the file samples.dtx.
%% 
%% This generated file may be distributed as long as the
%% original source files, as listed above, are part of the
%% same distribution. (The sources need not necessarily be
%% in the same archive or directory.)
%%
%%
%% Commands for TeXCount
%TC:macro \cite [option:text,text]
%TC:macro \citep [option:text,text]
%TC:macro \citet [option:text,text]
%TC:envir table 0 1
%TC:envir table* 0 1
%TC:envir tabular [ignore] word
%TC:envir displaymath 0 word
%TC:envir math 0 word
%TC:envir comment 0 0
%%
%%
%% The first command in your LaTeX source must be the \documentclass
%% command.
%%
%% For submission and review of your manuscript please change the
%% command to \documentclass[manuscript, screen, review]{acmart}.
%%
%% When submitting camera ready or to TAPS, please change the command
%% to \documentclass[sigconf]{acmart} or whichever template is required
%% for your publication.
%%
%%
\documentclass[sigplan,screen,nonacm]{acmart}

%%
%% \BibTeX command to typeset BibTeX logo in the docs
\AtBeginDocument{%
  \providecommand\BibTeX{{%
    Bib\TeX}}}

%% Rights management information.  This information is sent to you
%% when you complete the rights form.  These commands have SAMPLE
%% values in them; it is your responsibility as an author to replace
%% the commands and values with those provided to you when you
%% complete the rights form.
%\setcopyright{acmcopyright}
%\copyrightyear{2018}
%\acmYear{2018}
%\acmDOI{XXXXXXX.XXXXXXX}

%% These commands are for a PROCEEDINGS abstract or paper.
%\acmConference[Conference acronym 'XX]{Make sure to enter the correct
%  conference title from your rights confirmation emai}{June 03--05,
%  2018}{Woodstock, NY}
%%%
%%%  Uncomment \acmBooktitle if the title of the proceedings is different
%%%  from ``Proceedings of ...''!
%%%
%%%\acmBooktitle{Woodstock '18: ACM Symposium on Neural Gaze Detection,
%%%  June 03--05, 2018, Woodstock, NY}
%\acmPrice{15.00}
%\acmISBN{978-1-4503-XXXX-X/18/06}


%%
%% Submission ID.
%% Use this when submitting an article to a sponsored event. You'll
%% receive a unique submission ID from the organizers
%% of the event, and this ID should be used as the parameter to this command.
%%\acmSubmissionID{123-A56-BU3}

%%
%% For managing citations, it is recommended to use bibliography
%% files in BibTeX format.
%%
%% You can then either use BibTeX with the ACM-Reference-Format style,
%% or BibLaTeX with the acmnumeric or acmauthoryear sytles, that include
%% support for advanced citation of software artefact from the
%% biblatex-software package, also separately available on CTAN.
%%
%% Look at the sample-*-biblatex.tex files for templates showcasing
%% the biblatex styles.
%%

%%
%% The majority of ACM publications use numbered citations and
%% references.  The command \citestyle{authoryear} switches to the
%% "author year" style.
%%
%% If you are preparing content for an event
%% sponsored by ACM SIGGRAPH, you must use the "author year" style of
%% citations and references.
%% Uncommenting
%% the next command will enable that style.
%%\citestyle{acmauthoryear}


%%
%% end of the preamble, start of the body of the document source.
\begin{document}

%%
%% The "title" command has an optional parameter,
%% allowing the author to define a "short title" to be used in page headers.
\title{Secret Reviewer}

%%
%% The "author" command and its associated commands are used to define
%% the authors and their affiliations.
%% Of note is the shared affiliation of the first two authors, and the
%% "authornote" and "authornotemark" commands
%% used to denote shared contribution to the research.
\author{ChatGPT 4.0}
\email{webmaster@marysville-ohio.com}
%\affiliation{%
%  \institution{Institute for Clarity in Documentation}
%  \streetaddress{P.O. Box 1212}
%  \city{Dublin}
%  \state{Ohio}
%  \country{USA}
%  \postcode{43017-6221}
%}

%%
%% The abstract is a short summary of the work to be presented in the
%% article.
\begin{abstract}
	This document provides a concise set of rules for playing the board game "Secret Reviewer".
\end{abstract}


%%
%% Keywords. The author(s) should pick words that accurately describe
%% the work being presented. Separate the keywords with commas.
\keywords{Board game}
%% A "teaser" image appears between the author and affiliation
%% information and the body of the document, and typically spans the
%% page.
%\begin{teaserfigure}
%  \includegraphics[width=\textwidth]{sampleteaser}
%  \caption{Seattle Mariners at Spring Training, 2010.}
%  \Description{Enjoying the baseball game from the third-base
%  seats. Ichiro Suzuki preparing to bat.}
%  \label{fig:teaser}
%\end{teaserfigure}

\received{20 February 2007}
\received[revised]{12 March 2009}
\received[accepted]{5 June 2009}

%%
%% This command processes the author and affiliation and title
%% information and builds the first part of the formatted document.

\maketitle

\section*{Objective}

In the esteemed realm of academic research, scientists tirelessly work to publish their groundbreaking papers. But, some gatekeepers and the notorious "Reviewer 2" aim to reject these papers. Scientists must collaborate to publish their papers, while gatekeepers and "Reviewer 2" conspire to prevent this.

\section{Setup}

\begin{enumerate}
	\item \textbf{Publication Board}: Depending on the number of players, place the Publication Track on the table. This track will show the progress of papers, either toward publication or rejection.
	\item \textbf{Publication Deck}: Shuffle the criteria cards to form the Publication Deck. These cards decide the fate of the papers.
	\item \textbf{Roles Allocation}: Distribute the roles randomly amongst players. They can be Scientists, Gatekeepers, or the infamous Reviewer 2.
	\item Appoint the first Editor-in-Chief to kick off the game.
\end{enumerate}

\section{Role Revelation}

With all players closing their eyes, gatekeepers and "Reviewer 2" will silently acknowledge each other's identity. This clandestine recognition helps them in their covert mission against the scientists.

\section{Game Play}

\subsection{Selection}

\begin{enumerate}
	\item The Editor-in-Chief nominates an Associate Editor for this round.
	\item Players then cast a vote of confidence for this nomination. They can discuss, debate, and deduce before voting.
	\item If the majority favors the nomination, the game moves to the review phase. Otherwise, the Editorship moves to the next player.
\end{enumerate}

\subsection{Review Process}

\begin{enumerate}
	\item The Editor-in-Chief draws three criteria cards. After some contemplation, one card is discarded discreetly. The remaining two are passed to the Associate Editor.
	\item The Associate Editor, based on the given criteria, decides the paper's fate. More lenient criteria lead to publication; strict ones lead to rejection.
\end{enumerate}

\subsection{Editorial Powers}

As papers face rejection, the Editor-in-Chief gains special powers, escalating the game's intensity:

\begin{enumerate}
	\item \textbf{Request Clarifications}: Demand a player to clarify their stance on paper acceptance. This move can gather information or sow doubt.
	\item \textbf{Special Call for Papers}: Change the order and select the next Editor-in-Chief out of turn.
	\item \textbf{Direct Decision}: Decide on the paper without consulting the Associate Editor.
	\item \textbf{Choose Chief Reviewer}: A risky move. If "Reviewer 2" is chosen, the scientists' mission becomes daunting.
\end{enumerate}

\subsection{Veto Power}

After consecutive rejections, the Associate Editor can veto the criteria cards from the Editor-in-Chief. If both agree, discard the cards. Otherwise, the Associate Editor must accept one criteria.

\section{Winning the Game}

\begin{itemize}
	\item \textbf{Scientists}: Their victory lies in publishing a predefined number of papers.
	\item \textbf{Gatekeepers and Reviewer 2}: They triumph by preventing enough publications.
	\item An additional win for Gatekeepers: If they successfully install "Reviewer 2" as Chief Reviewer after many rejections.
\end{itemize}

\section*{Strategy}

The game demands a blend of strategy, intuition, and a pinch of luck. Will the scientists successfully navigate the treacherous waters of publication? Or will the gatekeepers, led by the dreaded "Reviewer 2," stifle their progress? Engage, deduce, and find out!

\clearpage

Second version, keep the best!

\section{Setup}

\begin{itemize}
	\item Select the Reviewer Track corresponding to the number of players and place it next to a Scientist Track. Note that every Reviewer track has an identical Scientist track on the back.
	\item Shuffle the 11 Gatekeeper policy mini-cards and the 6 Transparent mini-policy cards into a single policy deck. Place the deck face down on the DRAW card.
	\item Prepare an envelope for each player. Each envelope should contain a Secret Role card, the corresponding Research Group Membership card, and two Ballot cards. Use the table below to ensure the correct distribution of roles:
\end{itemize}

\begin{table}[h]
	\centering
	\begin{tabular}{|c|c|c|c|c|c|c|}
		\hline
		\textbf{Players\#} & 5 & 6 & 7 & 8 & 9 & 10 \\
		\hline
		Scientists & 3 & 4 & 4 & 5 & 5 & 6 \\
		\hline
		Gatekeepers & 1+R2 & 1+R2 & 2+R2 & 2+R2 & 3+R2 & 3+R2 \\
		\hline
	\end{tabular}
	\caption{Distribution of Roles based on Player Numbers}
	\label{tab:role_distribution}
\end{table}

After filling the envelopes, shuffle them to ensure each player's role remains secret. Each player should then randomly select one envelope.

Ensure the game has the correct number of ordinary gatekeepers, in addition to "Reviewer 2".

Following the role distribution, randomly select the first Editorial Lead and provide them both the Editorial Lead and Co-Editor placards. The process of ensuring players recognize their roles differs based on the number of players:

For games with 5-6 players:
\begin{enumerate}
	\item Ask everyone to close their eyes.
	\item Instruct the Gatekeepers and Reviewer 2 to open their eyes and acknowledge each other.
	\item Ask all players to close their eyes and put their hands down.
	\item Finally, allow everyone to open their eyes. If there are any issues or confusions, address them now.
\end{enumerate}

For games with 7-10 players:
\begin{enumerate}
	\item Ask everyone to close their eyes and form a fist in front of them.
	\item Instruct Reviewer 2 to keep eyes closed but extend their thumb.
	\item Gatekeepers (excluding Reviewer 2) should open their eyes and acknowledge each other. They should also note who has their thumb extended, identifying Reviewer 2.
	\item All players should then close their eyes and put their hands down.
	\item Finally, allow everyone to open their eyes. Address any confusions immediately.
\end{enumerate}

\section{Game Play}
"Secret Reviewer" is played in rounds, with each round consisting of:
\begin{itemize}
	\item Election to form an editorial board
	\item Review session to evaluate a new paper
	\item Editorial action to use publication power (if applicable)
\end{itemize}

\subsection{Election}

The primary aim is to choose an editorial board that will manage the review process effectively. This process consists of several steps:

\begin{enumerate}
	\item \textbf{Pass The Editorial Lead Candidacy:} Move the Editorial Lead placard clockwise to the next player at the start of a new round.
	\item \textbf{Nominate A Co-Editor:} The Editorial Lead Candidate nominates a Co-Editor Candidate by passing them the Co-Editor placard. Players involved in the evaluation of the last paper cannot be nominated as a Co-Editor.
	\item \textbf{Vote On The Board:} Everyone votes on the proposed editorial board. All players, including the candidates, should cast their vote.
	\item \textbf{Election Tracker:} If three consecutive board proposals are rejected, reveal and accept the top paper from the Paper Draw Deck. All players become eligible for the position of Co-Editor in the next election. Reset the tracker after a paper is accepted or a board is elected.
\end{enumerate}

\subsection{Review Session}

The Editorial Lead draws the top three papers from the Paper deck, rejects one, and then passes the remaining two to the Co-Editor. The Co-Editor rejects one and publishes the other. Any communication between the Editorial Lead and Co-Editor during this phase is prohibited.

\subsection{Editorial Action}

The Editorial Lead must act on any granted power upon the acceptance of a Gatekeeper-approved paper.

\subsection{Editorial Powers}

These powers include:
\begin{itemize}
	\item \textbf{Check Affiliation:} The Editorial Lead may check another player's Research Group Membership Card.
	\item \textbf{Special Issue:} The Editorial Lead can choose the next Editorial Lead Candidate.
	\item \textbf{Pre-review Peek:} The Editorial Lead can secretly view the top three papers.
	\item \textbf{Reject Paper:} The Editorial Lead can outright reject a paper. If Reviewer 2's favored paper is rejected, Scientists win.
	\item \textbf{Veto Power:} After five Gatekeeper-approved papers are accepted, the Co-Editor can veto a submission.
\end{itemize}


%%
%% The next two lines define the bibliography style to be used, and
%% the bibliography file.
\bibliographystyle{ACM-Reference-Format}
\bibliography{sample-base}


\end{document}
\endinput
%%
%% End of file `sample-sigplan.tex'.
