%%
%% This is file `sample-sigplan.tex',
%% generated with the docstrip utility.
%%
%% The original source files were:
%%
%% samples.dtx  (with options: `sigplan')
%% 
%% IMPORTANT NOTICE:
%% 
%% For the copyright see the source file.
%% 
%% Any modified versions of this file must be renamed
%% with new filenames distinct from sample-sigplan.tex.
%% 
%% For distribution of the original source see the terms
%% for copying and modification in the file samples.dtx.
%% 
%% This generated file may be distributed as long as the
%% original source files, as listed above, are part of the
%% same distribution. (The sources need not necessarily be
%% in the same archive or directory.)
%%
%%
%% Commands for TeXCount
%TC:macro \cite [option:text,text]
%TC:macro \citep [option:text,text]
%TC:macro \citet [option:text,text]
%TC:envir table 0 1
%TC:envir table* 0 1
%TC:envir tabular [ignore] word
%TC:envir displaymath 0 word
%TC:envir math 0 word
%TC:envir comment 0 0
%%
%%
%% The first command in your LaTeX source must be the \documentclass
%% command.
%%
%% For submission and review of your manuscript please change the
%% command to \documentclass[manuscript, screen, review]{acmart}.
%%
%% When submitting camera ready or to TAPS, please change the command
%% to \documentclass[sigconf]{acmart} or whichever template is required
%% for your publication.
%%
%%
\documentclass[sigplan,screen,nonacm]{acmart}

%%
%% \BibTeX command to typeset BibTeX logo in the docs
\AtBeginDocument{%
  \providecommand\BibTeX{{%
    Bib\TeX}}}

%% Rights management information.  This information is sent to you
%% when you complete the rights form.  These commands have SAMPLE
%% values in them; it is your responsibility as an author to replace
%% the commands and values with those provided to you when you
%% complete the rights form.
%\setcopyright{acmcopyright}
%\copyrightyear{2018}
%\acmYear{2018}
%\acmDOI{XXXXXXX.XXXXXXX}

%% These commands are for a PROCEEDINGS abstract or paper.
%\acmConference[Conference acronym 'XX]{Make sure to enter the correct
%  conference title from your rights confirmation emai}{June 03--05,
%  2018}{Woodstock, NY}
%%%
%%%  Uncomment \acmBooktitle if the title of the proceedings is different
%%%  from ``Proceedings of ...''!
%%%
%%%\acmBooktitle{Woodstock '18: ACM Symposium on Neural Gaze Detection,
%%%  June 03--05, 2018, Woodstock, NY}
%\acmPrice{15.00}
%\acmISBN{978-1-4503-XXXX-X/18/06}


%%
%% Submission ID.
%% Use this when submitting an article to a sponsored event. You'll
%% receive a unique submission ID from the organizers
%% of the event, and this ID should be used as the parameter to this command.
%%\acmSubmissionID{123-A56-BU3}

%%
%% For managing citations, it is recommended to use bibliography
%% files in BibTeX format.
%%
%% You can then either use BibTeX with the ACM-Reference-Format style,
%% or BibLaTeX with the acmnumeric or acmauthoryear sytles, that include
%% support for advanced citation of software artefact from the
%% biblatex-software package, also separately available on CTAN.
%%
%% Look at the sample-*-biblatex.tex files for templates showcasing
%% the biblatex styles.
%%

%%
%% The majority of ACM publications use numbered citations and
%% references.  The command \citestyle{authoryear} switches to the
%% "author year" style.
%%
%% If you are preparing content for an event
%% sponsored by ACM SIGGRAPH, you must use the "author year" style of
%% citations and references.
%% Uncommenting
%% the next command will enable that style.
%%\citestyle{acmauthoryear}
\usepackage{booktabs} % For better looking tables
\usepackage{siunitx} % For centering text in columns
\usepackage{tabularx} % For tables that exceed width of the page
\usepackage{todonotes}

%%
%% end of the preamble, start of the body of the document source.
\begin{document}

%%
%% The "title" command has an optional parameter,
%% allowing the author to define a "short title" to be used in page headers.
\title{Secret Reviewer}

%%
%% The "author" command and its associated commands are used to define
%% the authors and their affiliations.
%% Of note is the shared affiliation of the first two authors, and the
%% "authornote" and "authornotemark" commands
%% used to denote shared contribution to the research.
\author{ChatGPT 4.0}

\email{support@openai.com}
\affiliation{%
  \institution{OpenAI}
  \city{San Francisco}
  \state{California}
  \country{USA}
}

\author{DALL-E 2}

\email{support@openai.com}
\affiliation{%
	\institution{OpenAI}
	\city{San Francisco}
	\state{California}
	\country{USA}
}

\author{Ashwin Prasad S. Venkatesh \and Jonas Klauke \and Stefan Schott}

%\email{ashwin.prasad@upb.de}
%\email{jonas.klauke@upb.de}
%\email{stefan.schott@upb.de}
\affiliation{%
	\city{Paderborn}
	\country{Germany}
}

%\author{Jonas Klauke}
%
%\email{ashwin.prasad@upb.de}
%\email{jonas.klauke@upb.de}
%\email{stefan.schott@upb.de}
%\affiliation{%
	%	\city{Paderborn}
	%	\country{Germany}
	%}
%
%\author{Stefan Schott}
%
%\email{ashwin.prasad@upb.de}
%\email{jonas.klauke@upb.de}
%\email{stefan.schott@upb.de}
%\affiliation{%
	%	\city{Paderborn}
	%	\country{Germany}
	%}

%%
%% The abstract is a short summary of the work to be presented in the
%% article.
\begin{abstract}
	This document provides a concise set of rules for playing the board game ``Secret Reviewer''.
\end{abstract}


%%
%% Keywords. The author(s) should pick words that accurately describe
%% the work being presented. Separate the keywords with commas.
\keywords{Board game}
%% A "teaser" image appears between the author and affiliation
%% information and the body of the document, and typically spans the
%% page.
%\begin{teaserfigure}
%  \includegraphics[width=\textwidth]{sampleteaser}
%  \caption{Seattle Mariners at Spring Training, 2010.}
%  \Description{Enjoying the baseball game from the third-base
%  seats. Ichiro Suzuki preparing to bat.}
%  \label{fig:teaser}
%\end{teaserfigure}

%\received{20 February 2007}
%\received[revised]{12 March 2009}
%\received[accepted]{5 June 2009}

%%
%% This command processes the author and affiliation and title
%% information and builds the first part of the formatted document.

\maketitle

\section*{Objective}

In the esteemed realm of academic research, scientists tirelessly work to publish their groundbreaking papers. But, some bad reviewers and the notorious ``Reviewer~2'' aim to reject these papers. Scientists must collaborate to publish their papers, while bad reviewers and ``Reviewer~2'' conspire to prevent this.

\section{Setup}

\begin{enumerate}
	\item \textbf{Publication Board}: Depending on the number of players (c.f., Table \ref{tab:role_distribution}), place the Publication Track on the table. This track will show the progress of papers, either toward publication or rejection.
	\item \textbf{Criteria Deck}: Shuffle the criteria cards to form the Publication Deck. These cards decide the fate of the papers.
	\item \textbf{Roles Allocation}: Distribute the roles randomly amongst players. They can be \textit{accepting reviewers} (AR), \textit{rejecting reviewers} (RR), or the infamous \textit{Reviewer~2} (R2).
	\item Appoint the first \textit{general chair} to kick off the game.
\end{enumerate}

	\begin{table}[h]
%	\renewcommand{\arraystretch}{1.2}
	\centering
	\begin{tabularx}{.48\textwidth}{ccccccc}
		\toprule
		\textbf{\# Players} & \textbf{5} & \textbf{6} & \textbf{7} & \textbf{8} & \textbf{9} & \textbf{10} \\
		\midrule
		AR & 3 & 4 & 4 & 5 & 5 & 6 \\
		\midrule
		RR & 1+R2 & 1+R2 & 2+R2 & 2+R2 & 3+R2 & 3+R2 \\
		\bottomrule
	\end{tabularx}

	\caption{Distribution of Roles based on Player Numbers}
	\label{tab:role_distribution}
\end{table}

%\begin{table}[h]
%	\centering
%	\begin{tabularx}{|c|c|c|c|c|c|c|}
%		\hline
%		\textbf{Players \#} & 5 & 6 & 7 & 8 & 9 & 10 \\
%		\hline
%		AR & 3 & 4 & 4 & 5 & 5 & 6 \\
%		\hline
%		RR & 1+R2 & 1+R2 & 2+R2 & 2+R2 & 3+R2 & 3+R2 \\
%		\hline
%	\end{tabularx}
%	\caption{Distribution of Roles based on Player Numbers}
%	\label{tab:role_distribution}
%\end{table}


\section{Role Revelation}

Following the role distribution, randomly select the first \textit{general chair} and provide them both the general chair and \textit{program chair} placards.
The process of ensuring players recognize their roles differs based on the number of players:

\textbf{For games with 5-6 players:}
\begin{enumerate}
	\item Ask everyone to close their eyes.
	\item Instruct the rejecting reviewers and Reviewer~2 to open their eyes and acknowledge each other.
	\item Ask all players to close their eyes and put their hands down.
	\item Finally, allow everyone to open their eyes. If there are any issues or confusions, address them now.
\end{enumerate}

\textbf{For games with 7-10 players:}
\begin{enumerate}
	\item Ask everyone to close their eyes and form a fist in front of them.
	\item Instruct Reviewer~2 to keep eyes closed but extend their thumb.
	\item Rejecting reviewers (excluding Reviewer~2) should open their eyes and acknowledge each other. They should also note who has their thumb extended, identifying Reviewer~2.
	\item All players should then close their eyes and put their hands down.
	\item Finally, allow everyone to open their eyes. Address any confusions immediately.
\end{enumerate}

\section{Game Play}

\subsection{Selection}

\begin{enumerate}
	\item The GC nominates a PC chair for this round.
	\item Players then cast a vote of confidence for this nomination. They can discuss, debate, and deduce before voting.
	\item If the majority favors the nomination, the game moves to the review phase. Otherwise, the GC moves to the next player.
\end{enumerate}

\subsection{Review Process}

\begin{enumerate}
	\item The GC draws three criteria cards. After some contemplation, one card is discarded discreetly. The remaining two are passed to the PC chair.
	\item The PC chair, based on the given criteria, decides the paper's fate.
\end{enumerate}

\subsection{Powers of General Chair}

As papers face rejection, the GC gains special powers, escalating the game's intensity:

\begin{enumerate}
	\item \textbf{Request Clarifications}: Demand a player to clarify their stance on paper acceptance. This move can gather information or sow doubt.
	\item \textbf{Special Appointment}: Change the order and select the next GC out of turn.
	\item \textbf{Direct Decision}: Decide on the paper without consulting the PC chair.
	\item \textbf{Choose Chief Reviewer}: A risky move. If ``Reviewer~2'' is chosen, the accepting reviewers mission becomes daunting.
\end{enumerate}

\subsection{Veto Power}

After consecutive rejections, the PC chair can veto the criteria cards from the GC. If both agree, discard the cards. Otherwise, the PC chair must accept one criteria.

\section{Winning the Game}

\begin{itemize}
	\item \textbf{Accepting Reviewers}: Their victory lies in publishing a predefined number of papers.
	\item \textbf{Rejecting Reviewers and Reviewer~2}: They win by preventing enough publications.
	\item An additional win for rejecting reviewers: If they install ``Reviewer~2'' as GC after many rejections.
\end{itemize}

\section*{Strategy}

The game demands a blend of strategy, intuition, and a pinch of luck. Will the accepting reviewers successfully navigate the uphill battle of publications? Or will the rejecting reviewers, led by the dreaded ``Reviewer~2'' stifle their progress? Engage, deduce, and find out!


\section*{Acknowledgment}
This game is based on \href{https://www.secrethitler.com/}{SecretHiter}.

\section*{Availability}
The source is available on \href{https://github.com/ashwinprasadme/SecretReviewer}{GitHub}. Published under the same CC BY-NC-SA 4.0 license as the original game.

%%
%% The next two lines define the bibliography style to be used, and
%% the bibliography file.
%\bibliographystyle{ACM-Reference-Format}
%\bibliography{sample-base}


\end{document}
\endinput
%%
%% End of file `sample-sigplan.tex'.
